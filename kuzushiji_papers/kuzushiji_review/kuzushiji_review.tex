\documentclass[11pt]{article}
\usepackage{./preamble}
\usepackage{subfiles}

% Settings kinda things
\newcommand{\name}{Juniper Overbeck} % Hopefully this doesn't need to change again
\newcommand{\ttle}{Kuzushiji Review} % Title goes here

\begin{document}
\begin{itemize}
    \item \underline{RNN Based Uyghur Text Line Recognition and its Training Strategy:} In retrospect, this paper
        should never have made it onto the list. Mine it for the references it used, but \textit{never} cite a paper
        like this because of the obvious ethical problems. This paper is bad, its authors should feel bad, and they
        should be stricken from academia for their contribution to the racism against the Uyghur people in China.
    \item \underline{A Human-Inspired Recognition System for Pre-Modern Japanese Historical Documents:} This is
        highly relevant and should be one of the first papers that we take a serious look at, it uses the types of
        models we're interested in, is one of the most current papers on this exact classification task, and it is
        performed with the assistance of Japanese Literature Researchers, which should yield good domain knowledge.
        \begin{itemize}
            \item \underline{Classifier:} LSTM
            \item \underline{Accuracy:} $9.87\%$ to $53.81\%$ SER (sequence error rate)
                on the Level 2 and 3 CODH "Pattern Recognition and Media Understanding (PRMU) Algorithm Contest for
                Kuzushiji recognition" dataset. Not ideal for our cases.
            \item \underline{Relation to dataset:} Same language and macro classification task
            \item \underline{Technical Novelty:} Uses attention based/inspired recognition system that
                attempts to mimic how the human eye pays attention to the writing, in order to establish
                which parts of the art are actually the Kuzushiji text.
            \item \underline{Problems:} This is focused on the sequence error improvement, which conflicts
                with the dataset provided for the Kaggle competition, as the Kaggle competition is focused on the
                individual character recognition and placement. Maybe there's something here, but the big
                takeaway is to look for the CODH Kuzushiji classification contest, and see if there's anything
                fruitful in papers associated with the contest.
        \end{itemize}
\end{itemize}
\end{document}
